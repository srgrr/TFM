\section{Introduction}
\label{sec:introduction}
Most modern research fields use computational resources in some way or another. This trend started some decades ago, and it seems to be increasingly accepted and adopted among all the scientific fields. While computer science has brought many solutions to many research problems, it also caused a lot of difficulties and created new problems to researchers. One of the most remarkable problems is the increasing difficulty on programming and software development due to the increasing sizes of data sets, experiments, and number and complexity of the available computational resources. This problem becomes even more noticeable when a research group has no computer scientists and all the programming tasks are done by non-experts.\\
\\
\textit{Non-expert} programming has always been an issue, but it was far more manageable when all the programs ran sequentially in a single core machine. Multi-core CPUs allowed a program to run various fragments of its code at the same time and thus increased the difficulty of programming both in a conceptual and in a technical way. Parallel programming introduced concepts such as \textit{race condition}, \textit{data dependency}, \textit{critical region}, \textit{parallelism factor}, \textit{granularity}, \textit{overhead} and many more, and the existing frameworks (e.g: native threads) were too low level to be understood or used by non computer scientists, and implied high development and maintainment costs.\\
\\
This shift towards more complex computational models and programs due to the presence of parallelism encouraged the industry to create easier frameworks and programming models for parallel computing. One of the greatest examples of these new frameworks is OpenMP\cite{openmp08}. OpenMP simplified the task of writing parallel programs a lot, making it understandable to non computer scientists. As an example, a working sequential matrix multiplication algorithm can be written as follows:
\inputminted{c}{snippets/matmul_openmp.cc}
Note that all the forks, joins, private copies and similar are just \textit{specified}, but not done explicitly. This simplification allowed the general public to take advantage of parallelism. There are many other concurrent and parallel frameworks and models, such as MPI\cite{Forum:1994:MMI:898758}, and many programming languages, such as Java, have a built-in threading library, which is usually simpler to use than native, low-level threads. Even some languages, such as Erlang, are explicitly designed for concurrent and parallel programming.\\
\\
For many years the computational \textit{growth model} consisted of simply adding more resources to increase the potential degree of parallelism, as the performance improvements of individual processors stagnated. The parallel growth model also started to show signs of stagnation, but the demand of computational resources and the size of the problems to solve never stopped to increase. In fact, the sizes of the datasets experimented an exponential growth with the boom of paradigms such as Big Data or Deep Learning, in which it is not strange to deal with datasets that, simply put, do not fit in a single machine.\\
\\
So, the next step was obvious: make a program run in different machines simultaneously. This implied many new challenges, such as having many different memories and therefore many different versions of the same object allocated in machines with possibly different architectures. Some other problems appeared, as the impossibility to know the exact order in which events happened in a single program, as different machines have different clocks \cite{Lamport}, making the task of debugging this kind of programs even harder. The previous problems from parallel programming were inherited or got even worse. Data dependencies, race conditions, and the importance of granularity are still there. Some algorithms are much harder to implement in a distributed fashion due to not having the whole input, but only pieces or chunks.\\
\\



%TODO: Finish this thing please