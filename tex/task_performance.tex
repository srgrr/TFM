\newpage
\section{Fixing tasks overheads in PyCOMPSs}
\label{sec:task_overhead}
This first section describes a performance fix in the COMPSs programming model. Its intention is to give a practical example of how important is to properly manage objects in distributed programming models.

\subsection{Problem description}
A COMPSs user reported via mailing list that his application showed a gradual performance degradation over time. The workflow of his application can be summarized as follows:

\inputminted{python}{applications/TASK_OVERHEAD/main.py}

The user was able to detect this performance degradation thanks to the tracing tools. A trace showing this issue can be found in figure \ref{fig:trace_task_degradation}

% TODO: ADD TRACE FIGURE

\subsection{Analysing and narrowing down the problem}
The trace from figure \ref{fig:trace_task_degradation} gives us the hint that there is something wrong about-
% TODO: COMPLETE THIS SECTION, EXPLAIN HOW WE DID ARRIVE TO IDOBJ

\subsection{Object identification and mapping in PyCOMPSs}
Now we know that the main source of this overhead comes from the \verb|get_object_id| function from the PyCOMPSs source code. Let's take a look at this function:

\inputminted{python}{snippets/get_object_id_old.py}

This function iterates over potentially all of the tracked objects just to get the identifier of some object (or to assign it one). This is done this way because an object needs to be hashable for being used as a key in a dictionary. Hashable usually means immutable, and no user can guarantee us that his objects will fulfill these properties. In fact, the programming model supports INOUT objects, which are, by nature, mutable objects. Some examples of mutable Python objects are \verb|[1, 2, 3, "hello"]|, \verb|object()| and \verb|numpy.random.rand(5)|.\\
\\
A possible fix may consist of using the \verb|id| function. This function returns us the logical memory address of the object. However, this can bring some undesired behaviors. Let's consider the following code snippet:

\inputminted{python}{snippets/id_breaks.py}

After the line \verb|y1 = master_routine(1)| PyCOMPSs will track the object \verb|x|. Lets assume that we are using \verb|id(obj)| as the key of the \verb|obj2id| dictionary. After this function PyCOMPSs will track the object \verb|x|, so if \verb|id(x)| $ = 0x123$ then PyCOMPSs will know that the object allocated at the memory address $0x123$ will be mapped to the file \verb|file_1|. However, we must note that no actual references to these objects are stored in the PyCOMPSs data structures, so \verb|x| will be deleted by the garbage collector. Therefore, it is possible that the object \verb|x| from the second call will use the same memory address from the previous one. This means that we can reuse the previous \verb|x| object, leading to a very weird error which may be difficult to isolate and fix. In fact, memory address reutilisation is very common in Python:

\begin{verbatim}
Python 2.7.15rc1 (default, Nov 12 2018, 14:31:15) 
[GCC 7.3.0] on linux2
Type "help", "copyright", "credits" or "license" for more information.
>>> x = object()
>>> id(x)
140404280975488
>>> del x
>>> x = object()
>>> id(x)
140404280975488
\end{verbatim}

