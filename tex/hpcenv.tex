\section{High Performance Computing Environments}
\label{sec:hpcenv}

\subsection{A Brief Introduction to Supercomputers}
\label{subsec:intro_sc}

\subsection{Supercomputers and Queue Systems}
\label{subsec:hpc_queues}
Most supercomputers have many concurrent users. All of these users want to use some of the resources of the supercomputer, and usually in a selfish manner. This situation creates a lot of conflicts between users, and even some unethical behaviors such as killing the processes of other users. Also, many benchmarks and experiments require no noise introduced by concurrent, unrelated processes running in the same machine, so resource exclusivity must be guaranteed in these cases.\\
\\
The most common solution to the two aforementioned problems is to divide the different nodes of a supercomputer into login nodes and computing nodes. When a user opens a session in some supercomputer he will \textit{land} into some login node. Computing nodes are unreachable or even not visible by regular users, and the only way to have access to them is to ask the system for resources and wait until the system lends them to the user. The most common implementation of this resource assignment mechanism is a queue system. A queue system processes all the requests from the users, gives them a priority as a function of various parameters and lends them the requested resources according to these priorities, as a process scheduler does with processes and processors. In our project we will use generic scripts to enqueue all of our experiments.\\
\\
Two of the most common queue systems are LSF \cite{zhou1992lsf} and SLURM \cite{yoo2003slurm}. All the experiments of this project will be done in the Mare Nostrum 4 supercomputer, which uses SLURM.\\
\\
Although SLURM has its own micro-language and instructions, such as \verb|srun|, and submissions scripts, most of the experiments done in this project will not need them, as we will have generic queueing scripts available to us. A generic queueing script is a script capable to work with various queue systems to generate the corresponding specific queueing scripts. In our case, our script will translate our orders into a bunch of \verb|srun| commands and similar.
% TODO: REFER TO ENQUEUE COMPSS IN COMPSS SECTION