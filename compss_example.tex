\newpage
\subsection{A Full Example}
\label{subsec:compss_example}
This section intends to give the reader a more or less extensive insight on what writing a COMPSs application is. We think that this section may help to \textit{materialize} concepts and will avoid to give this document an excessively abstract tone.\\
\\
Lets suppose that we want to approximate the value of $\pi$. For this purpose we have thought on a simple, randomized algorithm:
\begin{enumerate}
\item Generate $N$ random 2D points with coordinates between $-1$ and $1$
\item Consider the set of points $S$ within distance $1$ or less to the origin
\item Assume that $\frac{|S|}{N} = \frac{\pi}{4}$
\end{enumerate}

\begin{figure}[ht!]
\centering
\includegraphics[scale=0.5]{figures/circle_square.png}
\caption{A graphical representation of the random experiment. The square has side length $2$, so the circle has radius $1$, and therefore area $\pi$. In ratio terms, $\frac{\pi}{4}$ of the points belong to the circle}
\label{fig:circle_square}
\end{figure}

A more graphical explanation on why this works can be found in figure \ref{fig:circle_square}.\\
\\
We know a little bit of Python, so we have decided to implement this program in it. Basically, our small application will consist of a \verb|test_random_point| function that generates a random point and return $1$ if this point lies inside our circle, and $0$ otherwise. We will call this function $N$ times, and consider the proportion $\frac{|S|}{N}$ to be equal to $\frac{\pi}{4}$.
\inputminted{python}{applications/PI_SQUARE/sequential.py}
This code can be straightforward \textit{optimized} by transforming the \verb|test_random_point| function into a COMPSs task and syncing the results in the main procedure.
\inputminted{python}{applications/PI_SQUARE/pycompss_naive.py}
Although this may be a good approach to \textit{parallelize} this application we must note that we want to make it run in a distributed environment. The main difference we can appreciate is that a COMPSs task may run in a different machine than the master code, so some coordination between two processes in different machines and the transfer of potentially big amounts of data are necessary. In other words, the tradeoff between task granularity and performance is much more punishing in distributed computing than in single-machine parallel cases.\\
\\
Another important thing to note is that a distributed application can still exploit lower level parallelism in each of its tasks. In our case, we can transform our \verb|test_random_point| function into a \verb|test_random_points| procedure that generates and tests various random points at the same time.

\inputminted{python}{applications/PI_SQUARE/pycompss_vectorized.py}

This last approach is what we consider a well \textit{COMPSsfied} application: it has a reasonable task count and granularity, and it exploits various levels of parallelism at the same time. This application also delegates most of the work to \verb|numpy| procedures, which are mainly written in C++ and OpenMP. This aspect is also important in PyCOMPSs, as Python is, by nature, a very slow programming language and it should be only used as an orchestrator.

COMPSs is mainly designed to run in HPC environments. Most HPC machines integrate some sort of queue system to manage its resources among all the demanding users. Our previous example can be run as a job in a queue system with the following command:

\inputminted{bash}{applications/PI_SQUARE/run_mn4.sh}

The \verb|enqueue_compss| command refers to a generic queueing script (see section \ref{subsec:hpc_queues}) which translates our request to enqueue this COMPSs job to a specific queue system. Some of the most common parameters of a COMPSs job can be found in table \ref{table:compss_queue_param}.\\
\\
\begin{table}[ht!]
\centering
\begin{tabular}{|l|l|}
\hline
Argument name   & Description                                                                           \\ \hline
\verb|exec_time|      & Job time limit                                                                        \\ \hline
\verb|num_nodes|      & \begin{tabular}[c]{@{}l@{}}Number of computing\\ nodes\end{tabular}                   \\ \hline
\verb|cpus_per_node| & \begin{tabular}[c]{@{}l@{}}Number of cores per\\ computing node\end{tabular}          \\ \hline
\verb|constraints|     & \begin{tabular}[c]{@{}l@{}}Additional constraints\\ (e.g: highmem nodes)\end{tabular} \\ \hline
\end{tabular}
\caption{Some example configuration parameters of the queue system. These parameters are usually passed as flags to the enqueue\_compss script.}
\label{table:compss_queue_param}
\end{table}
